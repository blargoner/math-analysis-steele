% Notes and exercises from The Cauchy-Schwarz Master Class by Steele
% By John Peloquin
\documentclass[letterpaper,12pt]{article}
\usepackage{amsmath,amssymb,amsthm,enumitem,fourier}

\newcommand{\CS}{Cauchy-Schwarz}

\newcommand{\R}{\mathbb{R}}

\newcommand{\mult}{\cdot}
\newcommand{\dx}{\,dx}

\newcommand{\Biglp}[1]{\Bigl({#1}\Bigr)}
\newcommand{\abs}[1]{|{#1}|}
\newcommand{\Biglabs}[1]{\Bigl|{#1}\Bigr|}
\newcommand{\innerprod}[2]{\langle{#1},{#2}\rangle}
\newcommand{\norm}[1]{\lVert{#1}\rVert}

% Theorems
\theoremstyle{definition}
\newtheorem*{exer}{Exercise}

\theoremstyle{remark}
\newtheorem*{rmk}{Remark}

% Meta
\title{Notes and exercises from\\\textit{The Cauchy-Schwarz Master Class}}
\author{John Peloquin}
\date{}

\begin{document}
\maketitle

\section*{Introduction}
This document contains notes and exercises from~\cite{steele}.

\section*{Chapter~1}
\begin{exer}[1.1]
For any real numbers \(r_1,\ldots,r_n\),
\[\sum_k r_k\le\sqrt{n}\,\Biglp{\sum_k r_k^2}^{1/2}\tag{a}\]
and
\[\sum_k r_k\le\Biglp{\sum_k\abs{r_k}^{2/3}}^{1/2}\Biglp{\sum_k\abs{r_k}^{4/3}}^{1/2}\tag{b}\]
\end{exer}
\begin{proof}
Both inequalities are obtained from \CS,\footnote{This refers to the first inequality in~\cite{steele}, p.~1.} the first by taking \(a_k=1\) and \(b_k=r_k\) for all~\(k\) and the second by taking \(a_k=r_k^{1/3}\) and \(b_k=r_k^{2/3}\) for all~\(k\), recalling that \(r^2=\abs{r}^2\) for any real number~\(r\).
\end{proof}

\begin{exer}[1.2]
Suppose \(p_k\ge 0\) for \(1\le k\le n\) and \(p_1+\cdots+p_n=1\). If \(a_k\ge 0\), \(b_k\ge 0\), and \(1\le a_kb_k\) for \(1\le k\le n\), then
\[1\le\Biglp{\sum_k p_ka_k}\Biglp{\sum_k p_kb_k}\]
\end{exer}
\begin{proof}
\(1\le\sqrt{a_kb_k}\), so \(p_k\le p_k\sqrt{a_kb_k}=\sqrt{p_ka_k\vphantom{b_k}}\sqrt{p_kb_k}\). By \CS,
\[1=\sum_k p_k\le\sum_k(p_ka_k)^{1/2}(p_kb_k)^{1/2}\le\Biglp{\sum_k p_ka_k}^{1/2}\Biglp{\sum_k p_kb_k}^{1/2}\]
The result now follows from squaring both sides of the outer inequality.
\end{proof}

\begin{exer}[1.3]
For any real sequences \((a_k),(b_k),(c_k)\) of length~\(n\),
\[\Biglp{\sum_k a_kb_kc_k}^4\le\Biglp{\sum_k a_k^2}^2\sum_k b_k^4\sum_k c_k^4\tag{a}\]
and
\[\Biglp{\sum_k a_kb_kc_k}^2\le\sum_k a_k^2\sum_k b_k^2\sum_k c_k^2\tag{b}\]
\end{exer}
\begin{proof}
By \CS,
\[\Biglabs{\sum_k a_kb_kc_k}\le\Biglp{\sum_k a_k^2}^{1/2}\Biglp{\sum_k b_k^2c_k^2}^{1/2}\]
so
\[\Biglp{\sum_k a_kb_kc_k}^2\le\Biglp{\sum_k a_k^2}\Biglp{\sum_k b_k^2c_k^2}\tag{c}\]
By \CS\ again,
\[\sum_k b_k^2c_k^2\le\Biglp{\sum_k b_k^4}^{1/2}\Biglp{\sum_k c_k^4}^{1/2}\]
which together with~(c) implies
\[\Biglp{\sum_k a_kb_kc_k}^2\le\Biglp{\sum_k a_k^2}\Biglp{\sum_k b_k^4}^{1/2}\Biglp{\sum_k c_k^4}^{1/2}\]
which implies~(a). Also, clearly
\[\sum_k b_k^2c_k^2\le\Biglp{\sum_k b_k^2}\Biglp{\sum_k c_k^2}\]
which together with~(c) implies~(b).
\end{proof}

\begin{exer}[1.4]
For all \(x,y,z>0\),
\[\left(\frac{x+y}{x+y+z}\right)^{1/2}+\left(\frac{x+z}{x+y+z}\right)^{1/2}+\left(\frac{y+z}{x+y+z}\right)^{1/2}\le 6^{1/2}\tag{a}\]
and
\[x+y+z\le 2\left(\frac{x^2}{y+z}+\frac{y^2}{x+z}+\frac{z^2}{x+y}\right)\tag{b}\]
\end{exer}
\begin{proof}
Both inequalities are obtained from \CS, the first by taking \(a_1=(x+y)^{1/2}\), \(a_2=(x+z)^{1/2}\), \(a_3=(y+z)^{1/2}\), and \(b_1=b_2=b_3=(x+y+z)^{-1/2}\), and the second by first writing
\[x+y+z=\frac{x}{\sqrt{y+z}}\sqrt{y+z}+\frac{y}{\sqrt{x+z}}\sqrt{x+z}+\frac{z}{\sqrt{x+y}}\sqrt{x+y}\]
to obtain
\begin{align*}
(x+y+z)^2\le2\left(\frac{x^2}{y+z}+\frac{y^2}{x+z}+\frac{z^2}{x+y}\right)(x+y+z)
\end{align*}
from which the result follows.
\end{proof}

\begin{exer}[1.5]
If \(p_k\ge 0\) for \(1\le k\le n\) and \(p_1+\cdots+p_n=1\), then
\[g(x)=\sum_k p_k\cos(\beta_k x)\qquad\text{satisfies}\qquad g^2(x)\le\frac{1}{2}\left(1+g(2x)\right)\]
\end{exer}
\begin{proof}
By \CS\ and the identity \(\cos^2(\alpha)=(1+\cos(2\alpha))/2\),
\begin{align*}
g^2(x)&=\Biglp{\sum_k\sqrt{p_k}\sqrt{p_k}\cos(\beta_k x)}^2\\
	&\le\Biglp{\sum_k p_k}\Biglp{\sum_k p_k\cos^2(\beta_k x)}\\
	&=\sum_k p_k\cos^2(\beta_k x)\\
	&=\frac{1}{2}\sum_k p_k\left(1+\cos(2\beta_k x)\right)\\
	&=\frac{1}{2}\left(1+g(2x)\right)\qedhere
\end{align*}
\end{proof}

\begin{exer}[1.6]
If \(p_k>0\) for \(1\le k\le n\) and \(p_1+\cdots+p_n=1\), then
\[\sum_k\left(p_k+\frac{1}{p_k}\right)^2\ge n^3+2n+\frac{1}{n}\]
Moreover, equality holds if and only if \(p_k=1/n\) for all~\(k\).
\end{exer}
\begin{proof}
By \CS,
\[n^2=\left(\sum_k\sqrt{p_k}\ \frac{1}{\sqrt{p_k}}\right)^2\le\Biglp{\sum_k p_k}\Biglp{\sum_k\frac{1}{p_k}}=\sum_k\frac{1}{p_k}\tag{a}\]
so
\[n^2+1\le\sum_k\frac{1}{p_k}+\sum_k p_k=\sum_k\left(p_k+\frac{1}{p_k}\right)\]
and by applying Exercise~1.1(a) to the last sum,
\[\frac{(n^2+1)^2}{n}\le\sum_k\left(p_k+\frac{1}{p_k}\right)^2\tag{b}\]
which is just the desired inequality. If equality holds in~(b), then equality holds in~(a), so \(p_k=1/n\) for all~\(k\) by~(1.11); the converse is trivial.
\end{proof}

\begin{exer}[1.7]
For any real numbers \(\alpha,\beta,x,y\),
\[(5\alpha x+\alpha y+\beta x+3\beta y)^2\le(5\alpha^2+2\alpha\beta+3\beta^2)(5x^2+2xy+3y^2)\]
\end{exer}
\begin{proof}
In~\(\R^2\), the product
\[\innerprod{(x_1,y_1)}{(x_2,y_2)}=5x_1x_2+x_1y_2+x_2y_1+3y_1y_2\]
is clearly symmetric and bilinear. Moreover,
\[\norm{(x,y)}^2=\innerprod{(x,y)}{(x,y)}=5x^2+2xy+3y^2\]
Fixing~\(y\) and considering the quadratic polynomial \(p(x)=5x^2+(2y)x+3y^2\), note \(p(x)\ge 0\) since \(5>0\) and \((2y)^2-4\mult 5\mult 3y^2=-56y^2\le 0\), and in fact \(p(x)=0\) if and only if \(x=y=0\). Therefore the product is positive definite and so an inner product. The desired inequality is now just
\[\innerprod{(\alpha,\beta)}{(x,y)}^2\le\norm{(\alpha,\beta)}^2\norm{(x,y)}^2\]
which is \CS~(1.16).
\end{proof}

\begin{exer}[1.8]
The following inequalities hold:
\begin{align*}
\sum_{k=0}^{\infty}a_k x^k&\le\frac{1}{\sqrt{1-x^2}}\,\Biglp{\sum_{k=0}^{\infty}a_k^2}^{1/2}&&(0\le x<1)\tag{a}\\
\sum_{k=1}^n\frac{a_k}{k}&<\sqrt{2}\,\Biglp{\sum_{k=0}^n a_k^2}^{1/2}&&\tag{b}\\
\sum_{k=1}^n\frac{a_k}{\sqrt{n+k}}&<(\log{2})^{1/2}\Biglp{\sum_{k=0}^n a_k^2}^{1/2}&&\tag{c}\\
\sum_{k=0}^n\binom{n}{k}a_k&\le\binom{2n}{n}^{1/2}\Biglp{\sum_{k=0}^n a_k^2}^{1/2}&&\tag{d}
\end{align*}
\end{exer}
\begin{proof}
The inequalities all follow from \CS, together with suitable bounds for~\(\sum_k b_k^2\).
For~(a), note \(0\le x^2<1\), so \(\sum_k (x^k)^2=\sum_k (x^2)^k=(1-x^2)^{-1}\) by convergence of the geometric series. For~(b), recall
\[\sum_{k=1}^n\frac{1}{k^2}<\sum_{k=1}^{\infty}\frac{1}{k^2}=\frac{\pi^2}{6}<2\]
For~(c), note
\[\sum_{k=1}^n\frac{1}{n+k}<\int_n^{2n}\frac{1}{x}\dx=\log(2n)-\log(n)=\log(2n/n)=\log(2)\]
For~(d), recall for any \(0\le m\le n\),
\[\binom{n}{k}=\sum_{j=0}^k\binom{m}{j}\binom{n-m}{k-j}\]
so
\[\binom{2n}{n}=\sum_{k=0}^n\binom{n}{k}\binom{n}{n-k}=\sum_{k=0}^n\binom{n}{k}^2\qedhere\]
\end{proof}

% References
\begin{thebibliography}{0}
\bibitem{steele} Steele, J.~Michael. \textit{The Cauchy-Schwarz Master Class.} Cambridge, 2004.
\end{thebibliography}
\end{document}
